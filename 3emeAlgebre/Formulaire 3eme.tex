\documentclass[a4paper,margin=1cm,fontscale=1]{baposter} % echelle de la police (initial à 0.285)

\usepackage[utf8]{inputenc}
\usepackage[T1]{fontenc}
\usepackage[french]{babel}
\frenchbsetup{StandardLists=true}
\usepackage{graphicx}
\usepackage{graphics}
\usepackage{epstopdf}
\usepackage{amsmath}
\usepackage{amssymb}
\usepackage{xlop}
\usepackage{booktabs}
\usepackage{pifont}
\usepackage{enumitem}
\usepackage{tabularx}
\usepackage{palatino}
\usepackage[font=small,labelfont=bf]{caption}
\usepackage{multicol}
\setlength{\columnsep}{5mm} % espace entre les colonnes à l'intérieur des cadres
\setlength{\columnseprule}{0mm} % largeur du filet entre les colonnes
\usepackage{pstricks}
\usepackage{pst-plot}
\usepackage{pstricks-add}

\newcommand{\Syst}[2]{\left\{\begin{array}{l} #1\\ #2 \end{array}\right.}
\setitemize{label=\ding{71},leftmargin=5mm,itemsep=0mm,parsep=0mm,topsep=0mm}
\renewcommand{\baselinestretch}{1.25}
\newcommand{\mili}[4]{\psgrid[subgriddiv=10, gridlabels=0, gridwidth=0.4pt, subgridwidth=0.4pt,gridcolor=brown!80,subgridcolor=brown!40](#1,#2)(#3,#4)}
\newcolumntype{C}[1]{>{\centering\arraybackslash}p{#1cm}}
\def\N{{\mathbb N}}
\def\Z{{\mathbb Z}}
\def\R{{\mathbb R}}
\def\C{{\mathbb C}}
\def\e{{\text{e}}}
\newcommand{\V}{\overrightarrow} % Vecteurs
\newcommand{\Repc}{(O;\V{u};\V{v})} % Repère Ouv
\newcommand{\Coor}[2]{\begin{pmatrix} #1\\#2 \end{pmatrix}} % Coordonnées
\newcommand{\Ree}{\mathfrak{Re}} % partie réelle
\newcommand{\Ima}{\mathfrak{Im}} % partie imaginaire
\newcommand{\Abs}[1]{\left \lvert#1\right \rvert} % module

\graphicspath{{Images/}}

\begin{document}


\begin{poster}
{
headerborder=open, % Ajout d'une bordure autour de l'entête des boites 
background=none, 
%bgColorOne=green, % Couleur de fond pour le dégradé sur le côté gauche de l'affiche
%bgColorTwo=yellow, % Couleur de fond pour le dégradé sur le côté droit de l'affiche
colspacing=6mm, % Espace entre les boites
columns=6,
borderColor=black, % Couleur des bordures de cadres
headerColorOne=white, % Couleur de fond pour l'en-tête dans les boites (côté gauche)
headerColorTwo=gray!50, % Couleur de fond pour l'en-tête dans les boites (côté droit)
headerFontColor=black, % Couleur du texte de l'en-tête des boites
boxColorOne=white, % Couleur de fond des boites
textborder=roundedleft, % Format des bordures : none, bars, coils, triangles, rectangle, rounded, roundedsmall, roundedright, faded
eyecatcher=true, % false pour supprimer le logo de gauche
headerheight=0.075\textheight, % Hauteur de l'en-tête
headershape=roundedright, % Format des bordures de l'entête : rectangle, small-rounded, roundedright, roundedleft or rounded
headerfont=\textsc\bf,% Police de l'en-tête : large, bold et sans serif %\textsc, 
textfont={\setlength{\parindent}{0em}}, % Indentation
linewidth=1pt % Largeur du filet des boites boxes
}
%----------------------------------------------------------------------
%	EN-TÊTE
%----------------------------------------------------------------------
{\mbox{}\hspace*{0.5cm}%\includegraphics[height=2cm]{Complexe.eps}
} %gauche
{\mbox{}{\Huge Formulaire brevet}}{\footnotesize 3eme} %titre
{%\includegraphics[height=2cm]{Logo_Nath.eps}\hspace*{0.5cm}\mbox{}
}%droite
% 

%---------------------
% Fiche
%---------------------

\headerbox{\bf Calcul littéral: notations} %%%%%%%
{name=cl_n,column=1,span=4}
{
%\begin{dotlist}
{$1x$ se note $x$}
\hfill{$-1x$ se note $-x$}
\hfill{$x$x$x$ se note $x^{2}$}
\hfill{$0x = 0$}

}

\headerbox{\bf Calcul littéral: développer et factoriser} %%%%%%%
{name=cl_df,column=0,span=4,below=cl_n}
{
\underline{\bf Développer}
\begin{itemize}
	\item \textbf{Développer}, c'est transformer un \underline{produit} en une \underline{somme} algébrique
	\item On peut utiliser les règles de la \textbf{distributivité}:
	\\$a(b+c) = ab + ac$
	\\$a(b-c) = ab - ac$
	\\$(a+b)(c+d)= ac + ad + bc + bd$
\end{itemize}
\underline{\bf Factoriser}
\begin{itemize}
	\item \textbf{Factoriser}, c'est transformer une \underline{somme} algébrique en un \underline{produit}.
	\item On peut utiliser les règles de distributivité:
	\\ $ab + ac =a(b+c)$
	\\ $ab - ac = a(b-c)$
\end{itemize}
} 

\headerbox{\bf Calcul littéral: équations} %%%%%%%
{name=cl_e,column=4,span=2,below=cl_n,bottomaligned=cl_df}
{
\begin{itemize}
   \item On peut effectuer la même opération sur les deux membres d'une équation
\end{itemize}
Exemple:
\\ $3x + 5 = -7$ {donne succesivement:}
\\ $(3x + 5) -5 = (-7) -5$ (on soustrait 5 dans chaque membre),soit $3x = -12$
\\ $ (3x)\div 3 = (-12) \div 3$ (on divise par 3 dans chaque membre), soit $x=-4$.
\\ \textbf{La solution de l'équation est $-4$.} 

}

\headerbox{\bf Puissances} %%%%%%%
{name=pow,column=0,span=6, below=cl_df}
{
\begin{itemize}
  \item Soit n un entier, $n>1$ et $a$ un nombre:\\
		$a^n = \underbrace{a \times  a \times  a \times  ... \times  a} _\textrm{$n$ facteurs $a$}$
		\hfill et si $a\neq 0$ , $ a^{-n} = \frac{1}{a^n}$\hfill {}
		\\ $a^1 = a$
		\hfill et, si $a \neq 0, a^0 = 1 $
		\hfill $a^{-1} = \frac{1}{a}$: c'est l'inverse de $a$
	\item Soit a un nombre, b un nombre $b \neq 0$, $n$ et $m$ deux entiers relatifs:
		\\ $a^n \times  a^m = a^{n+m}$
		\hfill $\frac{b^m}{b^n} = b^{m-n}$
		\hfill $(a^m)^n = a^{m\times n}$
		\hfill $ a^m\times b^m = (ab)^m$
		\hfill $ \frac{a^m}{b^m} = (\frac{a}{b})^m$
\end{itemize}
}

\headerbox{\bf Notation scientifique} 
{name=sci,column=0,span=3,below=pow}
{
On appelle notation scientifique d'un nombre strictement positif, une notation de la forme $a\times 10^n$, avec $1 \leq a < 10$
}

\headerbox{\bf Démonstration} %%%%%%%
{name=demo,column=3,span=3,below=pow}
{
Une \textbf{démonstration} consiste à expliquer de manière rigoureuse pourquoi un résultat est vrai. La démonstration comporte trois parties:
	\\1) Le théorème, la définition ou la propriété de la forme:
	\\Si \textbf{conditions} alors \textbf{conclusion}
	\\2) Vous devez recopier les données de votre exercice qui vérifient les conditions
	\\3) Vous pouvez donc appliquer la conclusion à votre problème
}

\headerbox{\bf Fractions} %%%%%%%
{name=frac,column=0,span=3,below=sci}
{
Pour additionner ou soustraires des fractions, on commence par les réduire au \textbf{\underline{même dénominateur}}.
\begin{itemize}
	\item \textbf {Somme}
	\\ $\frac{a}{b} + \frac{c}{b} = \frac{a+c}{b}$
	\item \textbf {Différence}
	\\ $\frac{a}{b} - \frac{c}{b} = \frac{a-c}{b}$
	\item \textbf {Simplification}
	\\ $\frac{a\times k}{b\times k} = \frac{a}{b}$
	\item \textbf {Multiplication}
	\\ $\frac{a}{b} \times  \frac{c}{d} = \frac{a\times c}{b\times d}$
	
	\item \textbf {Division}
	\\ $\frac{a}{b} \div \frac{c}{d} = \frac{a}{b} \times \frac{d}{c}$
	\\ L'inverse de $\frac{c}{d}$ est $\frac{d}{c}$
\end{itemize}
}

\headerbox{\bf Pourcentages} %%%%%%%
{name=per,column=3,span=3,below=demo}
{
\begin{itemize}
	\item \textbf{Augmenter de $a$\%} une valeur revient à la multiplier par $(1 + \frac{a}{100})$
	\item \textbf{Diminuer de $a$\%} une valeur revient à la multiplier par $(1 - \frac{a}{100})$
\end{itemize}
}


\end{poster}
\end{document}




